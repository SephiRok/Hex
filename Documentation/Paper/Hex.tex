\documentclass[a4paper, oneside, 12pt]{report}
\usepackage[slovene]{babel}
\usepackage[utf8]{inputenc}
\usepackage[left=3cm, top=3cm, right=2.5cm, bottom=2.5cm]{geometry}
\usepackage{fancyhdr}
\usepackage{etoolbox}
\usepackage{graphicx}
\usepackage{tabulary}
\usepackage{mathtools}

% TOC.
\usepackage[nottoc]{tocbibind}
\usepackage{tocloft}
\tocloftpagestyle{fancy}

% Rename things.
\addto\captionsslovene{
	\renewcommand{\contentsname}{Kazalo vsebine}
	\renewcommand{\listfigurename}{Kazalo slik}
	\renewcommand{\listtablename}{Kazalo tabel}
%	\renewcommand{\listappendicesname}{Kazalo prilog}
}

% Change chapter formatting.
\makeatletter
\def\@makechapterhead#1{
	\vspace*{30\p@}
	{\noindent\Huge\bf\thechapter \hspace*{0.5cm} \parindent \z@ \raggedright \normalfont
		\interlinepenalty\@M
		\Huge \bfseries #1\par\nobreak
		\vskip 30\p@
	}
}
\def\@makeschapterhead#1{
	\vspace*{30\p@}
	{\noindent\Huge\bf \parindent \z@ \raggedright \normalfont
		\interlinepenalty\@M
		\Huge \bfseries #1\par\nobreak
		\vskip 30\p@
	}
}
\makeatother

% Priloge.
%\usepackage{appendix}
%\addto\captionsslovene{
%	\renewcommand\appendixname{Priloga}
%	\renewcommand\appendixpagename{Priloge}
%}

% Kazalo prilog.
%\usepackage{tocloft}
%\newcommand{\listappendicesname}{Priloge}
%\newlistof{appendices}{apc}{\listappendicesname}
%\renewcommand{\appendices}[1]{\addcontentsline{apc}{appendices}{#1}}

% Seznam kratic.
\usepackage{longtable}
\newcommand\nomenclature[2]{#1 & #2 \\}

% Backreference.
\usepackage[pageref]{backref}
\usepackage{url}
\def\backreftwosep{ in~}
\def\backreflastsep{ in~}
\renewcommand*{\backref}[1]{}
\renewcommand*{\backrefalt}[4]{
	\ifcase #1 {\it (Ni citirano.)}
	\or        {\it (Citirano na strani~#2.)}
	\else      {\it (Citirano na straneh~#2.)}
	\fi}

%% To Do
% Read Reinforcement Learning: A Survey.
% Read Deepming research paper.

%% Alternative Titles
% Strojno učenje: učenje iz interakcije
% Strojno učenje: učenje z interakcijo
% Strojno učenje: okrepitveno učenje
% Strojno učenje z interakcijo z okoljem
% Strojno učenje v neznanem okolju
% Strojno učenje s poizkušanjem
% Strojno učenje na podlagi izkušenj
% Machine learning from trial-and-error

\pagestyle{fancy}
\fontfamily{timesnewroman}
\linespread{1.25}

\lhead{\footnotesize Breulj R. Strojno učenje iz interakcije
\\Univerza na Primorskem, FAMNIT, 2014}
\rhead{\thepage}
\cfoot{}
\setlength{\headheight}{24pt}

\begin{document}
\begin{titlepage}
\begin{center}
\begin{large}
UNIVERZA NA PRIMORSKEM\\
FAKULTETA ZA MATEMATIKO, NARAVOSLOVJE IN\\
INFORMACIJSKE TEHNOLOGIJE\\[6cm]
\end{large}
\end{center}

\begin{center}
Zaključna naloga\\
\textbf{\large Strojno učenje iz interakcije}\\
Machine learning from interaction\\[6cm]
\end{center}

\noindent
Ime in priimek: Rok Breulj\\
Študijski program: Računalništvo in informatika\\
Mentor: doc. dr. Peter Rogelj\\

\vfill
\begin{center}
{\large Koper, Avgust 2014}
\end{center}
\end{titlepage}

\pagenumbering{roman}

\section*{Ključna dokumentacijska informacija}
\medskip
\begin{center}
\fbox{\parbox{\linewidth}{
\vspace{0.2cm}
\noindent
Ime in PRIIMEK:\vspace{0.5cm}\\
Naslov zaključne naloge:\vspace{0.5cm}\\
Kraj:\vspace{0.5cm}\\
Leto:\vspace{0.5cm}\\
Število listov: \hspace{2cm} Število slik: \hspace{2.6cm} Število tabel:\hspace{2cm}\vspace{0.5cm}\\
Število prilog: \hspace{1.9cm} Število strani prilog: \hspace{1cm} Število referenc:\vspace{0.5cm}\\
Mentor:\vspace{0.5cm}\\
Somentor:\vspace{0.5cm}\\
Ključne besede:\vspace{0.5cm}\\
Math.~Subj.~Class.~(2010):\vspace{0.5cm}\\
{\bf Izvleček:}\\
Izvleček predstavlja kratek, a jedrnat prikaz vsebine naloge. V največ 250 besedah nakažemo problem, metode, rezultate, ključne ugotovitve in njihov pomen.
\vspace{0.2cm}
}}
\end{center}
\newpage

\section*{Key words documentation}
\medskip
\begin{center}
\fbox{\parbox{\linewidth}{
\vspace{0.2cm}
\noindent
Name and SURNAME:\vspace{0.5cm}\\
Title of final project paper:\vspace{0.5cm}\\
Place:\vspace{0.5cm}\\
Year:\vspace{0.5cm}\\
Number of pages:\hspace{1.6cm} Number of figures:\hspace{2.2cm} Number of tables:\vspace{0.5cm}\\
Number of appendices:\hspace{0.6cm} Number of appendix pages:\hspace{0.8cm}Number of references:\vspace{0.5cm}\\
Mentor:\vspace{0.5cm}\\
Co-Mentor:\vspace{0.5cm}\\
Keywords:\vspace{0.5cm}\\
Math.~Subj.~Class.~(2010):\vspace{0.5cm}\\
{\bf Abstract:}
\vspace{0.2cm}
}}
\end{center}
\newpage

\section*{Zahvala}
%Preface? Foreword?
% I was looking for a way to learn without prior knowledge of the problem. A universal learner.
% Reinforcement learning is one way -- trial and error interaction with the environment
% Observation (of the correct approach) is another, faster way of getting it right
% Teaching is the fastest way -- supervised learning
%In 1959, Arthur Samuel defined machine learning as a “Field of study that gives computers the ability to learn without being explicitly programmed”
% No one single learning way is best for everything, teaching is generally the fastest, but exercise and experiments afterwards are needed to absorb and enforce the knowledge.
\newpage

\tableofcontents
\thispagestyle{fancy}
\newpage

\listoftables
\thispagestyle{fancy}
\newpage

\listoffigures
\thispagestyle{fancy}
\newpage

%\listofappendices
%\thispagestyle{fancy}
%\addcontentsline{toc}{chapter}{Seznam prilog}
%\newpage

\chapter*{Seznam kratic}
\addcontentsline{toc}{chapter}{Seznam kratic}
\thispagestyle{fancy}
\begin{longtable}{@{}p{1cm}@{}p{\dimexpr\textwidth-1cm\relax}@{}}
\nomenclature{$tj.$}{to je}
\nomenclature{$npr.$}{na primer}
\end{longtable}
\newpage

\pagenumbering{arabic}

\chapter{Uvod}
\thispagestyle{fancy}
% learning, interaction, no teacher, environment, trial and error, influence through behavior, environment responds (input), intelligence, draw conclusions
Učimo se skozi naše celotno življenje. Eden izmed osnovnih načinov učenja temelji na podlagi interakcije z okoljem. V računalništvu pogosto radi odidemo po tej eksperimentalni poti, posebej ko verjamemo, da smo blizu rešitvi. Vendar pa se ni potrebno zazreti tako daleč, kot je računalništvo. Že kot otroci, ko mahamo z rokami in gledamo naokoli, nimamo izrecnega učitelja, imamo pa neposredno senzomotorično povezavo z našo okolico. S svojim vedenjem vplivamo na okolje in naša čutila izkoriščamo za pridobitev ogromne količine podatkov o vzrokih in učinkih, o posledicah dejanj in načinih, kako doseči cilje. Skozi naše življenje predstavljajo tovrstne interakcije velik vir znanja o našem okolju in o nas samih. Ko se učimo voziti avto ali pogovarjati, se zavedamo kako se okolje odziva na naša dejanja in iščemo način kako vplivati na rezultat z našim vedenjem.
%Ideja učenja iz interakcije z našim okoljem je ena od prvih, ki nam pride na misel, ko razmišljamo o naravi učenja. Ko se dojenček igra, maha z rokami ali gleda naokoli nima izrecnega učitelja, ima pa neposredno senzomotorično povezavo z okoljem. Uporaba te povezave proizvede ogromno informacije o vzrokih in učinkih, o posledicah dejanj in načinih kako doseči cilje. Skozi naše življenje so takšne interakcije nedvoumno velik izvir znanja o našem okolju in samih sebe. Ko se učimo voziti avto ali pogovarjati, se zavedamo kako se okolje odziva na naša dejanja in iščemo način kako vplivati na rezultat z našim vedenjem.~\cite{Intro}
%Learning from interaction is a foundational idea underlying nearly all theories of learning and intelligence.

% capable of surviving in a new environment, adapt an appropriate behavior, evaluate and classify new situations
% Intelligence measures an agent’s ability to achieve goals in a wide range of environments.” S. Legg and M. Hutter
Čeprav ni ene same standardne definicije inteligence, lahko primerjamo zbirko predlaganih definicij med seboj in hitro najdemo močne podobnosti med njimi. V veliko primerih, definicije inteligence vsebuje idejo, da se posameznik, ki je inteligenten, mora znati prilagoditi okoljem, ki jih ni še nikdar srečal, in v njih doseči cilje~\cite{ACollectionOfDefinitionsOfIntelligence}. Za inteligentno obnašanje očitno torej potrebujemo način, da ovrednotimo in razvrstimo nove položaje. Da se posameznik lahko uči in prilagodi svoje obnašanje, mora znati upoštevati tudi informacije iz okolja in iz njih sklepati. Okrepitveno učenje (angl. reinforcement learning) predstavlja teorijo o učenju povečanja nagrade na voljo v okolici in tako neposredno povečanje možnosti prilagoditve in preživetja. Nekatere naloge so preveč zapletene za jih opisati v statičnem računalniškem programu, kar je danes pogost postopek. Način za dinamično učenje in razvijanje programa je za nekatere naloge torej potreben.

% intelligent systems, dynamic real-world environments, autonomous, not relying on controlled conditions, uncertainty, time constrants, decision making, under these conditions reinforcement learning can have considerate advantages over other types of learning
Praktično vse kar živali, podjetja in računalniški programi delajo vključuje niz dejanj za dosego cilja. Ali je to vožnja avta do dela ali priprava jutranje kave, obstaja cilj in zaporedje dejanj za uspešno opravljen cilj. Prilagodljiv krmilni sistem, ki se zna učiti izvajati takšne sekvenčne naloge odločanja lahko najde vlogo v številnih domenah, kot so krmiljenje proizvodnega procesa, avtonomnih vozilih, letalstvu in pripomočkih za invalide. V pametnih sistemih, ki delujejo v dinamičnih okoljih resničnega sveta, kjer se ne moremo zanašati nad nadzorljivimi pogoji, kjer veljajo negotovost in časovne omejitve, ima lahko odločanje na podlagi okrepitvenega učenja obzirne prednosti pred ostalimi vrstami učenja.

% multi-disciplinary field: ai, psychology, control engineering, operations research, neuroscience, artificial neural networks, genetic algorithms
Področje okrepitvenega učenja je zelo interdisciplinarno, z močnimi vezami v teoriji krmiljenja (angl. control theory), psihologiji in nevroznanosti. Teorija krmiljenja pripomore k rešitvi problema z analitičnega, matematičnega vidika, med tem ko se psihologija in nevroznanost zgledujeta po bioloških procesih za odgovore. Veliko temeljnih smernic je izpeljanih iz psihologije vedenja in učenja; teorijah, ki se tičejo nagrajevanja in pogojevanja dejanj. Algoritmični pristopi so speljani pod podobnimi principi kot ljudje in živali oblikujemo vedenja glede na odzive iz okolice.

% past work
Zamisel o gradnji inteligentni strojev dolgo navdušuje človeštvo; že egipčani so o tem razmišljali. Čez leta se je razvilo veliko teorij, ampak komaj z uvodom modernega računalnika, 60 let nazaj, se je umetna inteligenca in strojno učenje razvilo v svojo znanstveno področje~\cite{TDANNForStrategicControlProblems}. Leta 1948 je Claude Elwood Shannon~\cite{AMathematicalTheoryOfCommunication} napisal predlog za šahovski program in leta 1959 je Arthur Samuel~\cite{SomeStudiesInMachineLearningUsingTheGameOfCheckers} razvil računalniški program, ki se je naučil igrati namizno igro dama z igranjem proti samemu sebi. V zadnjih letih so se raziskave osredotočale bolj na posnemanje bioloških modelov v poizkušanju izdelave programov, ki rešujejo probleme in razmišljajo kot ljudje. Nevrološke mreže (angl. neural networks), zelo poenostavljen model možganov, so bile uspešno uporabljene v vrsti aplikacijah. Po formalizaciji Samuelevega pristopa in oblikovanja učenja na podlagi časovne razlike lambda Richarda Suttona~\cite{LearningToPredictByTheMethodsOfTemporalDifference}, je v 1992 Richard Tesauro~\cite{PracticalIssuesInTemporalDifferenceLearning} razvil Backgammon igralca, ki je tekmoval proti najboljšim človeškim igralcem na svetu. Čeprav je Tesaurova združitev pristopa okrepitvenega učenja in nevroloških mrež pretresla področje umetne inteligence in Backgammon skupnosti, ni bilo veliko drugih uspehov v namiznih igrah~\cite{PlayingRiskAversiveGoOnALargeBoardUsingLocalNeuralNetworkPositionEvaluationFunctions, StrategyAcquisitionForTheGameOthelloBasedOnReinforcementLearning, LearningToEvaluateGoPositionsViaTemporalDifferenceMethods}. Prenos Tesaurove rešitve v največje namizne igre v področju umetne inteligence, šah in Go, niso uspele; rezultati so bili slabši kot so jih dosegle konvencionalne metode. Poleg namiznih iger je bilo okrepitveno učenje uporabljeno tudi v problemih robotike, razporejanja, dinamičnih dodelitev in optimizacije~\cite{ReinforcementLearningAnIntroduction}.

% 10000 hours to master skill, 20 hours to acquire new skill
% brain energy consumption and comparison to other primate and rodent brains -- we have the largest number of neurons in the cerebral cortex, and are the only ones that cook, allowing us to intake the energy needed to sustain the amount of neurons and the size of our body: https://www.youtube.com/watch?v=_7_XH1CBzGw
% Learning rate <-> human age.

% what's in this work
% computational approach, learning methods, ai, goal-directed learning from interaction, other approaches to machine learning
% reinforcement learning from the view of ai and engineering (not psychology or neuroscience)
% mathematical forumlation of learning from interaction
% unified notation
% how the different algorithms are related and combine, combinations 
% games are sequential decisions tasks, provide convenient testbeds for the study of such learning
% their goals and rules are well defined simplifying the modeling and simulation process
% at the same time, the problem they present are both challenging and interesting
% Hex
%Skozi implementacijo namizne igre Hex v nalogi raziščem okrepitveno učenje (angl. reinforcement learning), področje strojnega učenja, ki se ukvarja z vprašanjem kako se vesti v neznanem okolju, da bi povečali številčni nagrajevalni signal. Okrepitveno učenje se razlikuje od nadzorovanega učenja v tem, da nam niso nikoli prikazana pravilna dejanja ali pa napačna popravljena. Poudarek je tudi na zmogljivosti učenja med izvedbo, katero pelje do iskanja ravnovesja med raziskovanjem neznanih stanj in izkoriščanjem obstoječega znanja. Ukvarja se s celotnim problemom učenja iz interakcije z neznanim okoljem.
V nadaljevanju naloge je pregledan izvor okrepitvenega učenja iz vedenjske plati v razdelku~\ref{section:Psychology}. Nato je zavzet pogled iz strani umetne inteligence in inženirstva. Raziščen je računski pristop do učenja iz interakcije. V razdelku~\ref{section:} je matematično definiran celoten problem okrepitvenega učenja in v razdelku~\ref{section:} so predstavljene rešitve. Primerjani so različni algoritmi, njihove povezave in kombinacije. Ker imajo abstraktne namizne igre dobro definirane cilje in pravila, kar poenostavi model in simulacijo, in hkrati predstavljajo izziven in zanimiv problem, so priročno testno okolje za študijo takšnega učenja. 
V razdelku~\ref{} so rešitve iz okrepitvenega učenja uporabljene v namizni igri Hex in na koncu v razdelku~\ref{} razpravljeni rezultati skupaj s pogledom na prihodnost.
% cilj, originalnost dela je v aplikaciji okrepitvenega učenja na Hex igro

\section{Psihologija} \label{section:Psychology}
% https://new.edu/nodes/learning-and-behavior
% roots psychology of animal learning, where it takes its name
% dog clicker (=classical conditioning), dog training (=classical conditioning+operant for orders), people and animal learning
% psychology, operant conditioning (reinforcement), classical conditioning (association), (other non-related: observational learning)
% operant is for control, it affects the environment
% classical is for prediction, it does not affect the environment
% http://www.oxfordbibliographies.com/view/document/obo-9780199828340/obo-9780199828340-0043.xml
% Mazur, J. E. 2006. Learning and behavior. 6th ed. Upper Saddle River, NJ: Pearson Prentice Hall.
Okrepitveno učenje ima korenine v psihologiji učenja živali, iz kjer izvira tudi samo ime. Posebej se nanaša na klasično pogojevanje (angl. classical conditioning) in operantno pogojevaje (angl. operant conditioning).

\subsection{Klasično pogojevanje}
Klasično pogojevanje (imenovana tudi Pavlovo pogojevanje) je učenje prek povezav oz. asociacij.

V začetku 20. stoletja je ruski psiholog Ivan Pavlov (1849-1936) med preučevanjem prebavnega sistema psov odkril vedenjski fenomen~\cite{ConditionedReflexes}: psi so se začeli sliniti, ko so laboratorijski tehniki, ki so jih hranili, vstopili v sobo; čeprav psi še niso dobili hrane. Pavlov je spoznal, da so se psi začeli sliniti, ker so vedeli, da bodo nahranjeni; povezali so prihod tehnikov s hranjenjem.

S svojo ekipo je pričel raziskovati proces bolj podrobno. Izvedel je serijo eksperimentov v katerih so bili psi izpostavljeni zvoku trenutek preden so dobili hrano. Sistematično je nadzoroval časovno razliko med pojavom zvoka in hrano in zabeležil količino sline pri pseh. Najprej so se psi slinili samo, ko so videli ali zavohali hrano. Po večkratnem predstavljenem zvoku skupaj s hrano pa so se psi začeli sliniti takoj, ko so zaslišali zvok. Naučili so se povezave med zvokom in hrano, ki je sledila.

Pavlov je odkril temeljni asociativni proces imenovan klasično pogojevanje; učenje, ki se pojavi, ko nevtralna spodbuda (na primer: zvok) postane povezana s spodbudo, ki sama po sebi naravno proizvede vedenje (na primer: hrana). Po tem, ko je povezava naučena, predhodno nevtralna spodbuda zadošča za proizvedbo vedenja, ki je večinoma enakovredno (Pavlov je opazil razliko v sestavi sline~\cite{PavlovianConditioningItsNotWhatYouThinkItIs, LearningAndBehaviorAContemporarySynthesis, CognitionEvolutionAndBehavior}).

Prihod tehnikov oz. zvok je Pavlov imenoval pogojena spodbuda (angl. conditioned stimulus CS), ker je njen učinek odvisen od povezave s hrano. Hrano je imenoval nepogojena spodbuda (angl. unconditioned stimulus US), ker njen učinek ni odvisen od predhodnih izkušenj. Podobno je pogojen odziv (angl. conditioned response CR) odziv pogojene spodbude CS in nepogojen odziv (angl. unconditioned response UR) odziv nepogojene spodbude US. Pavlov je odkril, da je krajši razmak med zvokom in prikazom hrane povzročil močnejše in hitrejše učenje pogojenega odziva CR psa~\cite{PsychologyAStudentFriendlyApproach}.

% Make it black and white to fit the style of the rest?
\begin{figure}[htbp]
\includegraphics[scale=0.226]{WhistleAndDog.jpg}
\caption{Klasično pogojevanje piščalke namesto hrane za slinjenje pri psu~\cite{IntroductionToPsychology}.}
\label{figure:WhistleAndDog}
\end{figure}

Pogojevanje je evolucijsko koristno, ker omogoča organizmom razviti pričakovanja, ki jim pomagajo pri dobrih in slabih dogodkih. Razvidno je na primeru živali, ki zavoha novo hrano, jo poje in postane bolna. Če se žival zna naučiti povezave z vonjem (CS) in hrano (US), se bo znala izogibati določeni hrani že po vonju.

% After he had demonstrated that learning could occur through association, Pavlov moved on to study the variables that influenced the strength and the persistence of conditioning. In some studies, after the conditioning had taken place, Pavlov presented the sound repeatedly but without presenting the food afterward. Figure 7.4, “Acquisition, Extinction, and Spontaneous Recovery” shows what happened. As you can see, after the intial acquisition (learning) phase in which the conditioning occurred, when the CS was then presented alone, the behavior rapidly decreased—the dogs salivated less and less to the sound, and eventually the sound did not elicit salivation at all. Extinction refers to the reduction in responding that occurs when the conditioned stimulus is presented repeatedly without the unconditioned stimulus.

% Although at the end of the first extinction period the CS was no longer producing salivation, the effects of conditioning had not entirely disappeared. Pavlov found that, after a pause, sounding the tone again elicited salivation, although to a lesser extent than before extinction took place. The increase in responding to the CS following a pause after extinction is known as spontaneous recovery. When Pavlov again presented the CS alone, the behavior again showed extinction until it disappeared again.

% Although the behavior has disappeared, extinction is never complete. If conditioning is again attempted, the animal will learn the new associations much faster than it did the first time.

% Pavlov also experimented with presenting new stimuli that were similar, but not identical to, the original conditioned stimulus. For instance, if the dog had been conditioned to being scratched before the food arrived, the stimulus would be changed to being rubbed rather than scratched. He found that the dogs also salivated upon experiencing the similar stimulus, a process known as generalization. Generalization refers to the tendency to respond to stimuli that resemble the original conditioned stimulus. The ability to generalize has important evolutionary significance. If we eat some red berries and they make us sick, it would be a good idea to think twice before we eat some purple berries. Although the berries are not exactly the same, they nevertheless are similar and may have the same negative properties.

% The flip side of generalization is discrimination—the tendency to respond differently to stimuli that are similar but not identical. Pavlov’s dogs quickly learned, for example, to salivate when they heard the specific tone that had preceded food, but not upon hearing similar tones that had never been associated with food. Discrimination is also useful—if we do try the purple berries, and if they do not make us sick, we will be able to make the distinction in the future. And we can learn that although the two people in our class, Courtney and Sarah, may look a lot alike, they are nevertheless different people with different personalities.

% In some cases, an existing conditioned stimulus can serve as an unconditioned stimulus for a pairing with a new conditioned stimulus—a process known as second-order conditioning. In one of Pavlov’s studies, for instance, he first conditioned the dogs to salivate to a sound, and then repeatedly paired a new CS, a black square, with the sound. Eventually he found that the dogs would salivate at the sight of the black square alone, even though it had never been directly associated with the food. Secondary conditioners in everyday life include our attractions to things that stand for or remind us of something else, such as when we feel good on a Friday because it has become associated with the paycheck that we receive on that day, which itself is a conditioned stimulus for the pleasures that the paycheck buys us.

Klasično pogojevanje obravnava samo problem napovedovanja, ker odziv živali ne vpliva na eksperiment, oziroma, bolj splošno, ne vpliva na okolje. Učenje na podlagi časovne razlike (angl. temporal difference learning), opisano pozneje v razdelku~\ref{subsection:TD0Prediction}, je bilo prvotno predvsem povezano s klasičnim pogojevanjem in problemom napovedovanja, kjer pogojena spodbuda (CS) povezana s poznejšo nepogojeno spodbudo (US) povzroči potrebo po ovrednotenju časovne razlike vrednostne funkcije. Cilj izračuna je zagotoviti, da po učenju pogojena spodbuda (CS) postane napovednik nepogojene spodbude (US). Kratek osnutek na temo algoritmičnih pristopov do eksperimentov klasičnega pogojevanja dajeta Belkenius in Morén~\cite{ComputationalModelsOfClassicalConditioningAComparativeStudy}.

Čeprav je bilo učenje na podlagi časovne razlike prvotno zasnovano za reševanje problema napovedovanja, je uporabljeno tudi za rešitve problema optimalnega krmiljenja (glej razdelek~\ref{subsection:TD0Control})~\cite{ReinforcementLearningAnIntroduction}. %Zlasti omembe vredno je delo Watkinsa na Q-učenju (angl. Q-learning)~\cite{}, algoritem za krmiljenje na podlagi časovne razlike.

\subsection{Operantno pogojevanje}
V klasičnem pogojevanju se organizem nauči povezati nove spodbude z naravnim biološkim odzivom kot je slinjenje ali strah. Organizem sam se ne nauči nič novega, ampak začne izvajati že obstoječe vedenje v prisotnosti novega signala. Operantno pogojevanje, po drugi strani, je učenje, ki se zgodi glede na posledice vedenja in lahko vsebuje nova dejanja. Operantno pogojevanje je, ko se pes usede na ukaz, ker je bil pohvaljen za dejanje v preteklosti. Operantno pogojevanje je, ko nasilnež v šoli grozi sošolcem, ker mu to dovoli doseči svoje cilje in, ko otrok dobi dobre ocene, ker so mu starši zagrozili s kaznijo. Pri operantnem pogojevanju se organizem uči iz posledic svojih dejanj.

Psiholog Edward L. Thorndike (1874-1949) je bil prvi, ki je sistematično preučil operantno pogojevanje. Izdelal je škatlo, katero je mogoče odpreti samo ob rešitvi preproste uganke. V njo je spustil mačko in opazoval dogodke. Sprva so mačke praskale in grizle naključno. Ampak sčasoma so slučajno potisnile ročico in odprle vrata za katerimi je stala nagrada -- ostanki ribe. Naslednjič, ko je bila mačka zaprta v škatlo je poizkusila manjše število neučinkovitih dejanj preden se je uspešno osvobodila. Po več poizkusih se je mačka naučila skoraj takoj pravilno odzvati.~\cite{AnimalIntelligence1}

% Thorndike puzzle box drawing.

Z opazovanjem teh sprememb v vedenju mačke je pripeljalo psihologa Thorndike do razvitja njegovega zakona o učinku: princip, da se odzivi, ki tipično pripeljejo do prijetnega izida v določenem položaju, bolj verjetno pojavijo ponovno v podobnem položaju; med tem ko pa so odzivi, ki tipično pripeljejo do neprijetnega izida, manj verjetni, da se ponovno pojavijo v tem položaju.~\cite{AnimalIntelligence2}

Vedenjski psiholog B. F. Skinner (1904-1990) je razširil te ideje in jih povezal v bolj popoln sistem, ki opredeljuje operantno pogojevanje. Zasnoval je operantne komore (tako imenovane Skinner škatle) za sistemično preučevanje učenja; majhno zaprto strukturo, dovolj veliko za glodalca ali ptico, in z palico ali gumbom katerega je lahko žival pritisnila ali kljunila za nagrado vode ali hrane. Vsebovalo je tudi napravo za grafični zapis odzivov živali.~\cite{IntroductionToPsychology}

% Skinner box drawing.

Najbolj osnoven eksperiment je Skinner izvedel zelo podobno kot Thorndike z mačkami. Podgana spuščena v škatlo se je odzvala po pričakovanjih; hitela je okrog, vohljala in praskala po tleh in stenah. Čez čas je podgana slučajno naletela na gumb, ga pritisnila in dobila košček hrane. Naslednjič je potrebovala manj časa in z vsakim novim poizkusom je hitreje pritisnila gumb. Kmalu je pritiskala na gumb kolikor hitro je lahko jedla hrano. Kot pravi zakon o učinku, se je podgana naučila ponavljati dejanje, ki ji je pridobilo hrano in prenehala dejanja, ki niso.~\cite{IntroductionToPsychology}

Skinner je preučeval kako živali spreminjajo svoje vedenje v odvisnosti od okrepitve (angl. reinforcement) in kaznovanja (angl. punishment). Določil je izraze, ki razlagajo proces operantnega učenja (glej tabelo~\ref{table:OperantConditioningTerms}). Izraz okrepitev je poimenoval dogodek, ki utrdi ali zviša verjetnost nekega vedenja in kaznovanje dogodek, ki oslabi ali zniža verjetnost nekega vedenja. Uporabil je še izraze pozitivno in negativno za opredeliti, če je spodbuda predstavljena ali odvzeta. Pozitivna okrepitev torej utrdi odziv z predstavitvijo nečesa prijetnega in negativna okrepitev utrdi odziv z znižanjem ali odvzemom nečesa neprijetnega. Na primer, pohvala otroka za opravljeno domačo nalogo je pozitivna okrepitev med tem ko pa jemanje aspirina za zniževanje glavobola predstavlja negativno okrepitev. V obeh primerih okrepitev zviša verjetnost, da se bo vedenje ponovilo v prihodnosti.~\cite{IntroductionToPsychology}

% positive reinforcement, negative reinforcement
% positive punishment, negative punishment
% reinforcement = increase behavior
% punishment = decrease behavior
% positive = event response produces stimulus
% negative = event response removes stimulus
% Simply put, reinforcers serve to increase behaviors whereas punishers serve to decrease behaviors; thus, positive reinforcers are stimuli that the subject will work to attain, and negative reinforcers are stimuli that the subject will work to be rid of or to end.[9] The table below illustrates the adding and subtracting of stimuli (pleasant or aversive) in relation to reinforcement vs. punishment.
\begin{table}[htbp]
\begin{tabulary}{\textwidth}{| L | L | L | L |}
\hline
\textbf{Izraz} & \textbf{Opis} & \textbf{Izid} & \textbf{Primer} \\ \hline
Pozitivna okrepitev & Predstavljena ali povečana prijetna spodbuda & Vedenje je utrjeno & Otrok dobi slaščico po tem, ko pospravi sobe \\ \hline
Negativna okrepitev & Znižana ali odvzeta neprijetna spodbuda & Vedenje je utrjeno & Starši prenehajo pritoževati po tem, ko otrok pospravi sobo \\ \hline
Pozitivno kaznovanje & Predstavljena ali povečana neprijetna spodbuda & Vedenje je oslabljeno & Učenec dobi dodatno domačo nalogo po tem, ko nagaja v razredu \\ \hline
Negativno kaznovanje & Znižana ali odvzeta prijetna spodbuda & Vedenje je oslabljeno & Otrok izgubi privilegij računalnika po tem, ko pride pozno domov \\ \hline
\end{tabulary}
\caption{Vpliv pozitivne in negativne okrepitve in kaznovanja na vedenje.}
\label{table:OperantConditioningTerms}
\end{table}

Čeprav je razlika med okrepitvijo (povišanje verjetnosti vedenja) in in kaznovanjem (znižanje verjetnosti vedenja) je navadno jasno, je v nekaterih primerih težko določiti. če je pozitivno ali negativno. V vročem poletnem dnevu je lahko svež veter zaznan kot pozitivna okrepitev (ker prinese hladnejši zrak) ali pa negativna okrepitev (ker odvzame vroč zrak). V nekaterih primerih je lahko okrepitev hkrati pozitivna in negativna. Za odvisnika, jemanje drog hkrati prinese užitek (pozitivna okrepitev) in odstrani neprijetne simptome umika (negativna okrepitev).~\cite{IntroductionToPsychology}

Pomembno se je tudi zavedati da okrepitev in kaznovanje niso samo nasprotni. Uporaba pozitivne okrepitve za spremembo vedenja je skoraj vedno bolj učinkovito kot kaznovanje. To je zato, ker pozitivna okrepitev osebo ali žival spravi v boljšo voljo in pripomore k vzpostavitvi pozitivnega razmerja z osebo, ki predstavlja okrepitev. Tipi pozitivne okrepitve, ki so učinkoviti v vsakdanjem življenju vključujejo verbalne pohvale in odobritve, podelitev statusa in prestiža in direktno finančno izplačilo. Kaznovanje, po drugi strani, je bolj verjetno, da ustvari samo začasne spremembe v vedenju, ker temelji na prisili in vzpostavi negativno in kontradiktorno razmerje z osebo, ki predstavlja kazen. Ko se oseba, ki kazen predstavi, umakne iz okolja, se neželeno vedenje verjetno vrne.~\cite{IntroductionToPsychology}

Operantno pogojevanje je metoda učenja, ki stoji za izvedbo številnih trikov pri živalih. V filmih in predstavah so živali, od psov do konjev in delfinov, naučeni dejanj z uporabo pozitivnih okrepitev; skačejo čez ovire, se vrtijo, pomagajo osebi pri vsakdanjih opravilih in izvajajo še druga neobičajna dejanja.~\cite{IntroductionToPsychology}

Velikokrat se pri učenju hkrati izvaja klasično in operantno pogojevanje. Učitelji imajo s seboj napravo, ki proizvede specifičen zvok. Učenje se začne z nagrajevanjem želenega enostavnega dejanja s hrano (operantno pogojevanje) in hkrati s povezavo hrane z zvokom (klasično pogojevanje). Hrana je tako lahko intervalno izpuščena in pred dejanjem je dodan še zvočni ukaz na katerega želimo učeno dejanje (klasično pogojevanje). S tem povežemo samo zvočni ukaz pred dejanjem z nagrado hrane. Kompleksnejša dejanja so postopoma naučena iz enostavnejših z nadaljno povezavo spodbud, kar je imenovano proces oblikovanja.~\cite{IntroductionToPsychology}

Spodbude, ki so naravno zadovoljive organizmu, kot so hrana, vodi in izvzetje bolečine se imenujejo primarne spodbude, med tem, ko pa je sekundarna spodbuda neutralni dogodek, ki je povezan s primarno spodbudo s pomočjo klasičnega pogojevanja. Primer sekundarne spodbude je zvok, ki je povezan s primarno spodbudo, hrano. Dodaten primer vsakdanje sekundarne spodbude je denar. Radi imamo denar, vendar ne zaradi denarja po sebi, temveč zaradi primarnih spodbud, stvari, ki jih denar lahko kupi.~\cite{IntroductionToPsychology}

Tudi domači ljubljenčki se naučijo obnašanja na podlagi teh konceptov; in ne samo na ukaz, ampak tudi kako se vesti na povodcu, do tujcev itd. S to metodo je celo možno naučiti živali razlikovati med podobnimi vzorci, kar omogoča znanstvenikom preizkusiti sposobnost učenja pri živalih. Porter, Neuringer~\cite{MusicDiscriminationByPigeons} so, na primer naučili golobe razlikovati med stili glasbe in Watanabe, Sakamoto, Wakita~\cite{PigeonsDiscriminationOfPaintings} med stili umetnosti.

Operantno pogojevanje se razlikuje od klasičnega v tem, da spremeni vedenje do okolja. Ne obravnava več samo problema napovedovanja, ampak širši problem krmiljenja.

\chapter{Problem okrepitvenega učenja} \label{chapter:Problem}
\thispagestyle{fancy}
{\em Okrepitveno učenje (angl. reinforcement learning)} po~\cite{ReinforcementLearningAnIntroduction} je učenje kaj narediti, kako izbirati dejanje, da povečamo številčni nagrajevalni signal. Učencu niso nikoli predstavljena pravilna ali optimalna dejanja kot pri večini oblik strojnega učenja. Katera dejanja prinesejo največjo nagrado mora sam odkriti s poizkušanjem. Skozi interakcijo z okoljem se uči posledic svojih dejanj. V najbolj zanimivih in težavnih primerih imajo dejanja vpliv ne le na takojšnjo nagrado ampak tudi na naslednji položaj in posledične nagrade. Te dve karakteristiki, iskanje s poizkušanjem in zamudne nagrade, so dve najpomembnejši lastnosti okrepitvenega učenja.

%Okrepitveno učenje ni definirano s karakterističnimi metodami učenja temveč kot karakterizacija problema učenja. Katerokoli metodo primerno za rešitev problema smatramo kot metodo okrepitvenega učenja. Celoten problem okrepitvenega učenja je predstavljen na strani \pageref{chapter:Problem}. Osnovna ideja je zajeti najpomembnejše vidike realnega problema s katerim se sooča učenec (angl. learning agent) pri interakciji s svojim okoljem za dosego cilja. Takšen učenec mora imeti nekakšna čutila za pridobivanje informacij o stanju okolja in mora biti sposoben vplivati na to stanje z dejanji. Imeti mora tudi cilj ali pa cilje, ki se nanašajo na stanje okolja. Namen opisa problema je predstaviti te vidike, čutenje, dejanje in cilj, v najenostavnejši obliki brez poenostavljenja.

Okrepitveno učenje se razlikuje od nadzorovanega učenja (angl. supervised learning) v tem, da nima izobraženega zunanjega nadzornika, ki predloži učencu primere in rezultate. Nadzorovano učenje je pomemben tip učenja vendar ni primerno za učenje iz interakcije. V interaktivnih problemih je velikokrat nepraktično pridobiti primere želenega vedenja, ki so pravilni in hkrati predstavljajo vsa stanja v katerih mora učenec delovati. V neznanem okolju, kjer bi si lahko predstavljali, da je učenje najbolj koristno, se mora učenec učiti iz svojih izkušenj.

%Another key feature of reinforcement learning is that it explicitly considers the whole problem of a goal-directed agent interacting with an uncertain envi- ronment. This is in contrast with many approaches that consider subproblems without addressing how they might fit into a larger picture. For example, we have mentioned that much of machine learning research is concerned with supervised learning without explicitly specifying how such an ability would finally be useful. Other researchers have developed theories of planning with general goals, but without considering planning’s role in real-time decision- making, or the question of where the predictive models necessary for planning would come from. Although these approaches have yielded many useful results, their focus on isolated subproblems is a significant limitation.

%Reinforcement learning takes the opposite tack, starting with a complete, interactive, goal-seeking agent. All reinforcement learning agents have ex- plicit goals, can sense aspects of their environments, and can choose actions to influence their environments. Moreover, it is usually assumed from the beginning that the agent has to operate despite significant uncertainty about the environment it faces. When reinforcement learning involves planning, it has to address the interplay between planning and real-time action selection, as well as the question of how environmental models are acquired and im- proved. When reinforcement learning involves supervised learning, it does so for specific reasons that determine which capabilities are critical and which are not. For learning research to make progress, important subproblems have to be isolated and studied, but they should be subproblems that play clear roles in complete, interactive, goal-seeking agents, even if all the details of the complete agent cannot yet be filled in.

% Further, there is a focus on on-line performance, which involves finding a balance between exploration (of uncharted territory) and exploitation (of current knowledge).

To poglavje formalno definira dele okrepitvenega učenja ter določi predpostavke potrebne za opis rešitev v sledečih poglavjih.

\section{Elementi okrepitvenega učenja} \label{section:ReinforcementLearningElements}
%IS THIS THE ISSUE I HAD IN THE ALG?
%We use Rt+1 instead of Rt to denote the immediate reward due to the action taken at time t because it emphasizes that the next reward and the next state, St+1, are jointly determined.

% agent and the environment
% policy
% reward func
% value func
% model

%% agent
% learner, decision maker
% interacts with the environment to solve tasks, learning algorithms optimize his behavior, he learns
Cilj algoritmov okrepitvenega učenja je optimizirati vedenje {\em učenca (angl. agent)}. Učenec je tisti, ki se skozi interakcijo odloča o {\em dejanjih (angl. action)} za rešitev zadane {\em naloge (angl. task)}.

%% environment
% signal to the agent
% anything the agent cannot change arbitrarily is considered part of the environment
% reward computation is always external and part of the environment, the agent does not have absolute control over it
% More specifically, the agent and environment interact at each of a sequence ofdiscretetimesteps,t=0,1,2,3,....2 Ateachtimestept,theagentreceives some representation of the environment’s state, St ∈ S, where S is the set of possible states, and on that basis selects an action, At ∈ A(St), where A(St) is the set of actions available in state St. One time step later, in part as a consequence of its action, the agent receives a numerical reward , Rt+1 ∈ R, and finds itself in a new state, St+1.3 Figure 3.1 diagrams the agent–environment interaction.
% this is an abstract and flexible framework and applies to many problems
% time steps don't need to be fixed interval of time, actions could be low level controls, voltages, or high-level decisions like what school to go to or who to marry
Učenec se s svojimi dejanji vede na {\em okolje (angl. environment)}. Vse kar učenec ni zmožen poljubno spremeniti se smatra kot izven učenca in pripada okolju. Učenec in okolje neprestano vplivata na drug drugega; učenec izbira dejanja in okolje se odziva na ta dejanja s predstavitvijo novih {\em stanj (angl. state)} učencu. Okolje, ob prehodih na nova stanja, oddaja tudi številčne {\em nagrade (angl. reward)}, katere učenec poizkuša povečati skozi čas. Natančneje, učenec in okolje so v interakciji v vsakem koraku diskretnega zaporedja časa $t = 0, 1, 2, 3, \dots$. V vsakem koraku učenec prejme predstavitev stanja okolja, $s_t \in S$, kjer je $S$ množica vseh možnih stanj. Glede na stanje se odloči za dejanje, $a_t \in A(s)$, iz množice možnih dejanj. En časovni korak pozneje prejme učenec, kot posledica svojega dejanja, številčno nagrado, $r_{t+1} \in R$, in se znajde v novem stanju, $s_{t+1}$. Slika \ref{figure:AgentEnvironment} prikazuje opisan potek interakcije med učencem in okoljem. Takšna opredelitev elementov okrepitvenega učenja ustreza številnim težavam. Ni nujno, da časovni koraki predstavljajo fiksne intervale v resničnem času, lahko se nanašajo na poljubne zaporedne faze odločanja oziroma delovanja. Dejanja so lahko v zelo nizkem nivoju, na primer napetosti, ki krmilijo roko robota, ali pa odločitve v visokem nivoju, na primer v katero šolo se vpisati ali pa kakšno hrano pripraviti za večerjo. Stanja so lahko tudi v zelo različnih predstavitvah, od nizko-nivojskih odčitkov senzorjev do abstraktnih simboličnih opisov sob. V splošnem so lahko dejanja katerekoli odločitve o katerih se želimo učiti in stanja karkoli, ki nam lahko pomaga pri teh odločitvah. Edini cilj učenca je povečati prejete nagrade čez čas.

\begin{figure}[htbp]
\includegraphics[scale=1.0]{AgentEnvironment.png}
\caption{Interakcija med učencem in njegovim okoljem~\cite{ReinforcementLearningAnIntroduction}.}
\label{figure:AgentEnvironment}
\end{figure}

%% policy
% sufficient to define behavior
% instructs the agent what to do in each state
% maps EACH state to an action
% agent uses policy to choose the next action
% This mapping is called the agent’s policy and is denoted πt, where πt(a|s) is the probability that At = a if St = s.
% correspondts to psychology's stimulus-response rules or associacions
% may be stochastic
{\em Politiko (angl. policy)} $\pi$ imenujemo pravilo po katerem se učenec odloča katero dejanje izvesti v vsakem od stanj. Z drugimi besedami: politika preslikuje stanja v dejanja. Sama po sebi zadostuje za popolno definirano vedenje. V času $t$ predstavlja $\pi_t(a_t|s_t)$ verjetnost, da učenec izvede dejanje $a_t$ v stanju $s_t$. V psihologiji koncept politike ustreza povezavam spodbud z odzivi. V splošnem so politike lahko stohastične.

%% reward function
% defines the goal, sole objective is ot maximize the reward in the long run
% maps states to rewards, scalar values, Rt element of real numbers
% indicates how well the agent is doing
% positive encourages state visits, negative discourages
% defines what is good and bad events
% in biology, it could be pleasure and pain; positive/negative simulus(?)
% may be stochastic
{\em Nagrajevalna funkcija (angl. reward function)} opredeljuje cilje v nalogi okrepitvenega učenja, saj povečava nagrad čez čas predstavlja edini cilj učenca. V grobem, stanja okolja preslikuje v realno število, $r_t$, nagrado, ki predstavlja trenutno zaželenost stanja. Pozitivne nagrade spodbujajo obiske stanj, negativne pa jih odvračajo. Številne nagrade so pogosto preprosto definirane kot $+1$ ali $-1$. Oddane nagrade predstavljajo, kako dobro se učenec vede v okolju. Nagrade definirajo dobre in slabe dogodke. V biološkem sistemu bi lahko nagrade identificirali kot užitek ali bolečina. V psihologiji so to okrepitve ali kaznovanje. Nagrajevalna funkcija je nujno del okolja in je učenec ne sme biti zmožen spremeniti. V splošnem so nagrajevalne funkcije lahko stohastične.
% about how to choose +1, -1? seems limiting. From Intro 3.2 Goals and Rewards.

%% value function (V)
% reward function indicates what is good in an immediate sense, value function specifies what is good in the long run
% rl algorithms model this function
% maps states to values values; a state's desariability
% expresses expected reward (return?) gained from visiting the state, immediate and in the long run; the total amount an agent can expect to accumulate over the future, starting from a state
% takes into account states that are likely to follow and the rewards from those states
% make analogy to biology; long term reward
% rewards primary, values secondary, without rewards there would be no values, but we make decisions based on values
% rewards are given by the environment, values must be estimated and reestimated
% most important component of almost all rl algos is a method for efficiently estimating values
% the value function is specific to the policy
Med tem ko nagrajevalna funkcija označuje kaj je dobro v takojšnjem smislu, {\em vrednostna funkcija (angl. value function)} $V$ določa kaj je dobro na dolgi rok. Vrednostna funkcija izraža pričakovano nagrado iz obiska stanja, hkrati takojšnjo in dolgoročno; to je, izraža skupno količino nagrade, ki jo učenec lahko predvideva nabrati čez čas z začetkom v določenem stanju. Vrednosti upoštevajo stanja, ki verjetno sledijo, in nagrade iz teh stanj. Vrednostna funkcija $V$ preslikuje stanja $s$ v vrednosti $v$. Najpomembnejši del skoraj vseh algoritmov okrepitvenega učenja je učinkovito ocenjevanje vrednosti. Tudi v vsakdanjem življenju velikokrat ocenjujemo in napovedujemo dolgoročno vrednost situacij, kar nam predstavlja nemajhen izziv. Stanja, ki imajo visoko takojšnjo nagrado so lahko dolgoročno slaba in imajo nižjo vrednost kot alternativna. Slaščica, na primer, ima kratkoročen užitek, ampak velikokrat ni najboljša izbira prehrane kar se tiče našega telesnega zdravja. Obratno je tudi mogoče, stanje ima lahko zelo nizko takojšnjo nagrado, ampak se skozi prihodnost izkaže za najboljšo izbiro. Naša pot do službe je lahko hitrejša, če ne izberemo očitno najkrajšo pot, ampak upoštevamo promet do katerega nas zadana pot pripelje. Tudi živali se pri operantnem učenju učijo vrednosti in ne le takojšnjih nagrad. Hitro se naučijo, da dolgoročno prejmejo nagrade hitreje, če se pravilno vedejo, kot pa če ne. Nagrade so primarne med tem, ko so vrednosti sekundarne. Brez nagrad ne bi bilo vrednosti, ampak vedemo se glede na ocene vrednosti. Razlika je tudi v tem, da so nagrade oddane iz okolja, vrednosti pa moramo neprestano ocenjevati učenci sami. Vrednostna funkcija določa politiko vedenja, saj s svojim vedenjem želimo povečati nagrade katere vrednostna funkcija opisuje.

%% action value function (Q)
% same as for V, but maps states to actions
{\em Vrednostna funkcija dejanj (angl. action-value function)} $Q$ je enakovredna vrednostni funkciji, s to razliko, da stanja $s$ slika direktno v dejanja $a$. Vrednostna funckija $V$ določa vrednost $v$ stanju $s$ ($V(s) = v$) med tem, ko vrednostna funkcija dejanj določa vrednost $v$ dejanju $a$ ($Q(a) = v$).

%% optinal environment model
% mimics the behavior of the environment, for given state and action it predicts the next state and reward
% used for planning
% dynamic programing methods use models
% in some algorithms a model of the environment can be used, in others it's omitted
%Even if the agent has a complete and accurate environment model, the agent is typically unable to perform enough computation per time step to fully use it. The memory available is also an important constraint. Memory may be required to build up accurate approximations of value functions, policies, and models. In most cases of practical interest there are far more states than could possibly be entries in a table, and approximations must be made.
Nekateri algoritmi lahko uporabijo tudi model okolja, pri drugih je pa opuščen.

\section{Končen Markov proces odločanja}
% MARKOV REF?
V okrepitvenem učenju se učenec vede na podlagi signala iz okolja, ki ga imenujemo tudi stanje okolja. V tem razdelku je opisano kaj je zahtevano od signala stanja in kakšne informacije je smiselno od signala pričakovati.

Po eni strani lahko signal stanja pove veliko več, kot samo trenutne meritve. Stanja so lahko predstavljena z močno obdelanimi originalnimi meritvami, ali pa s kompleksnimi strukturami, ki so zgrajena skozi čas. Če slišimo odgovor ``da'' se znajdemo v zelo različnih stanjih odvisno od predhodnega vprašanja, ki ga ne slišimo več.

Po drugi strani pa ne smemo predpostavljati, da nam signal stanja zna povedati vse o okolju, ali celo vse kar nam bi prišlo prav za odločanje. Če igramo igro s kartami ne smemo predvidevati, da bomo izvedeli kaj imajo drugi igralci v rokah ali pa katera je naslednja karta na vrhu kupa. Če učenec odgovori na telefon, ne smemo predpostavljati, da ve kdo ga kliče vnaprej. V obeh primerih obstajajo skrite informacije stanja, ki jih učenec ne more vedeti, ker jih ni nikoli prejel.

Kar bi radi, idealno, je signal stanja, ki povzame vse uporabne predhodne informacije. Za to je ponavadi potrebna več kot samo trenutna informacija, ampak nikoli več kot celotna preteklost vseh prejetih informacij. Signal stanja, ki zadrži vse uporabne predhodne informacije, pravimo, da je {\em Markov}, oziroma, da ima {\em Markovo lastnost}. Na primer, pri igri štiri v vrsto je trenutna konfiguracija vseh polj Markovo stanje, ker povzame vse kar je pomembno o poteku igre. Čeprav je veliko informacije o poteku igre izgubljeno, je vse pomembno še vedno na voljo.

Pri končnem številu stanj in nagrad, je v splošnem dinamika okolja definirana samo s popolno porazdelitvijo verjetnosti
\begin{equation} \label{equation:GeneralCompleteProbabilityDistribution}
Pr\{r_{t+1} = r, s_{t+1} = s' | s_0, a_0, r_1, \dots, s_{t-1}, a_{t-1}, r_t, s_t, a_t\}
\end{equation}
na odziv okolja v času $t+1$, na dejanje v času $t$ in za vse vrednosti $r$, $s'$ in prejšnjih dogodkov $s_0, a_0, r_1, \dots, s_{t-1}, a_{t-1}, r_t, s_t, a_t$. Če ima signal stanja {\em Markovo lastnost}, pa je odziv okolja v času $t+1$ odvisen samo od stanja in dejanja v času $t$ in lahko dinamiko okolja definiramo z določitvijo le porazdelitve verjetnosti
\begin{equation} \label{equation:MarkovCompleteProbabilityDistribution}
Pr\{r_{t+1} = r, s_{t+1} = s' | s_t, a_t\},
\end{equation}
za vse $r$, $s'$, $s_t$ in $a_t$. Z drugimi besedami, signal stanja ima {\em Markovo lastnost}, in je {\em Markovo stanje}, če in samo če je (\ref{equation:GeneralCompleteProbabilityDistribution}) enako (\ref{equation:MarkovCompleteProbabilityDistribution}) za vse $s'$, $r$, in preteklosti $s_0, a_0, r_1, \dots, s_{t-1}, a_{t-1}, r_t, s_t, a_t$. V tem primeru pravimo, da ima celotno okolje in naloga Markovo lastnost.

Če ima okolje Markovo lastnost, lahko iz enostavne dinamike predhodnega stanja (\ref{equation:MarkovCompleteProbabilityDistribution}) napovedujemo naslednje stanje in naslednjo nagrado za trenutno stanje in dejanje. V tem okolju lahko napovedujemo vsa stanja in nagrade v prihodnosti enako dobro kot bi lahko s popolno preteklostjo do trenutnega časa. Sledi enako, da je najboljša politika izbire dejanj enako dobra v Markovem stanju kot najboljša politika izbire dejanj s trenutnim stanjem in popolno zgodovino.

Čeprav Markova lastnost velikokrat ne drži popolnoma v nalogah okrepitvenega učenja, je vseeno zelo primerno razmišljati o stanju v okrepitvenem učenju kot približnemu Markovemu stanju. Omogoča nam razmišljati o odločitvah in vrednostih na podlagi trenutnega stanja.

Naloga okrepitvenega učenja, ki zadovoljuje Markovo lastnost imenujemo {\em Markov proces odločanja (angl. MDP -- Markov decision process)}. Če je prostor stanj in dejanj končno, potem to imenujemo {\em končen Markov proces odločanja (angl. finite MDP)}.

Končen MDP je torej popolnoma definiran s:
\begin{itemize}
\item končno množico dosegljivih stanj $S$,
\item končno množico izvedljivih dejanj $A$,
\item prehodno funkcijo, definirano na vseh stanjih iz $S$ in za vsa dejanja iz $A$, ki je za prehod v stanje $s' \in S$ odvisna samo od trenutnega stanja $s \in S$ in dejanja $a \in A$: 
\begin{equation}
p(s' | s, a) = Pr\{s_{t+1} = s' | s_t = s, a_t = a\},
\end{equation}
\item nagrajevalno funkcijo, ki je, posledično od prehodne funkcije, tudi odvisna samo od trenutnega stanja in trenutnega dejanja:
\begin{equation}
r(s, a, s') = E[r_{t+1} | s_t = s, a_t = a, s_{t+1} = s'].
\end{equation}
\end{itemize}

\section{Diskretno okrepitveno učenje}
Ta razdelek povzame okrepitveno učenje \cite{ReinforcementLearningAnIntroduction} v diskretnem primeru, to je, v primeru, ko je prostor stanj okolja diskreten in končen in čas je razdeljen v diskretne korake.

Politika $\pi$ slika stanje v dejanje, kot omenjeno v razdelku \ref{section:ReinforcementLearningElements}. Za končne MDP lahko definiramo tudi {\em optimalno politiko (angl. optimal policy)} $\pi^\star$. Naj bosta $\pi$ in $\pi^\star$ politiki in $V^\pi$ vrednostna funkcija politike $\pi$ ter $V^{\pi^\star}$ vrednostna funkcija politike $\pi^\star$. Politika $\pi^\star$ je optimalna, če ima vrednostno funkcijo  $V^{\pi^\star}$ z naslednjo lastnostjo
\begin{equation}
V^{\pi^\star}(s) >= V^\pi(s), \forall s,
\end{equation}
za vse možne politike $\pi$.

% episodic vs continuing tasks, returns, unified notation
Kar je bilo do sedaj navedeno kot pričakovana dolgoročna nagrada je pogosto imenovan {\em pričakovan donos (angl. expected return)}. Formalna definicija donosa nekega stanja v času $t$ za poslednje nagrade $r_{t+1}, r_{t+2}, r_{t+3} \dots$ je
\begin{equation}
R_t = r_{t+1} + \gamma r_{t+2} + \gamma^2 r_{t+3} + \dots = \sum_{k=0}^{T-t-1} \gamma^k r_{t+k+1},
\end{equation}
kjer je $0 \le \gamma \le 1$. Donos je vsota vseh nadaljnjih nagrad, ki jih pričakujemo po času $t$ do končnega stanja v času $T$. V končnem stanju definiramo, da je prehod možen samo v isto stanje in nagrada ob prehodu vedno ničelna. S tem lahko poenotimo {\em epizodične} in {\em neskončne naloge} z uvedbo konstante $\gamma$, ki predstavlja {\em faktor popuščanja (angl. discount factor)}, s tem, da je lahko $T = \infty$ ali $\gamma = 1$, ampak ne oba hkrati. Pri neskončnih nalogah, ki jih ne moremo razdeliti na epizode, je $T = \infty$, saj se nikoli ne končajo, hkrati pa mora biti $\gamma < 1$, drugače lahko donos postane neskončen. Faktor popuščanja določa koliko želimo upoštevati prihodnje nagrade. Z $\gamma = 0$ se začne učenec brigati samo za trenutno nagrado. Pri epizodičnih nalogah, kot so igre, je navadno $\gamma = 1$.

Za končne MDP lahko {\em vrednost stanja $s$ po politiki $\pi$, t.i. vrednostno funkcijo $V^\pi(s)$}, definiramo kot pričakovan donos iz stanja $s$ z nadaljnjim upoštevanjem politike $\pi$, formalno:
\begin{equation}
V^\pi(s) = E_\pi[R_t | s_t = s] = E_\pi\Bigg[\sum_{k=0}^\infty \gamma^k r_{t+k+1} \Bigg| s_t = s\Bigg],
\end{equation}
kjer $E_\pi[.]$ predstavlja pričakovano vrednostjo, če učenec sledi politiki $\pi$.

Podobno lahko {\em vrednost dejanja $a$ v stanju $s$} po politiki $\pi$, t.i. vrednostno funkcijo dejanj $Q^\pi(s, a)$, definiramo kot pričakovan donos iz stanja $s$ ob dejanju $a$ in nadaljnjim upoštevanjem politike $\pi$, formalno:
\begin{equation}
Q^\pi(s, a) = E_\pi[R_t | s_t = s, a_t = a] = E_\pi\Bigg[\sum_{k=0}^\infty \gamma^k r_{t+k+1} \Bigg| s_t = s, a_t = a\Bigg].
\end{equation}

\section{Raziskovanje in izkoriščanje}
Eden od izzivov okrepitvenega učenja, ki jih ne najdemo v ostalih oblikah strojnega učenja, je kompromis med raziskovanjem (angl. exploration) in izkoriščanjem (angl. exploitation). Med učenjem, ko učenec uporablja približek optimalne vrednostne funkcije za svoje vedenje, mu to omogoča pridobiti največjo znano nagrado, ampak nikjer ni zagotovil, da je ta znana politika tudi v splošnem najboljša. Boljša rešitev bi mogoče lahko bila najdena, če bi učenec imel dovoljenje raziskovati dejanja, ki jih še ni poizkusil.

Spodaj opisana $\epsilon$-požrešna izbira dejanj je zelo preprosta in v praksi pogosto uporabljena. Obstaja še veliko drugih metod, nekaterih bolj kompleksnih od drugih. Več primerov je v delu Thrun \cite{TheRoleOfExplorationInLearningControl} ter Sutton in Barto \cite{ReinforcementLearningAnIntroduction}.

\subsection{$\epsilon$-požrešna izbira dejanj}
Ena izmed najbolj enostavnih pristopov k izbiri dejanja za ravnovesje med raziskovanjem in izkoriščanjem je uvod parametra $\epsilon$, ki določi verjetnost izbire naključnega dejanja. Na vsakem koraku po tej metodi učenec izbere naključno dejanje z verjetnostjo $\epsilon$ in požrešno dejanje z verjetnostjo $1 - \epsilon$.

Velikokrat je koristno izbrati veliko naključnih dejanj ob začetku učenja in nato kot učenje napreduje znižati pogostost naključnih dejanj. S tem v fazi največjega učenja čimbolj raziščemo prostor stanj. Znižanje vrednosti $\epsilon$ logično zniža stopnjo raziskovanja in zviša stopnjo izkoriščanja. Težava pri {\em $\epsilon$-požrešni metodi izbiranja dejanj (angl. $\epsilon$-greedy action selection}) je, da ne obstaja preprostega načina za izbiro vrednosti $\epsilon$. V veliko primerih je težavno izbrati kdaj povečati ali znižati število naključnih dejanj, ki naj jih učenec izbere.

%\section{Primeri}
%children playing with blocks
%matching shapes

% Reinforcement learning is about learning from interaction how to behave in order to achieve a goal.


%1. Prediction only: RL is used to learn the value function for the policy followed. At the end of learning this value function describes for every visited state how much future reward we can expect when performing actions starting at this state.
%2. Control: By means of RL, we wish to find that particular set of policies which maximizes the reward when travelling through state space. This way we have at the end obtained an optimal policy which allows for action planning and optimal control.
%1Thus, RL also assumes that such systems “follow the Markov property”. Essentially this means that it is unimportant along which path a certain state has been reached. Once there, the state itself contains all relevant information for future calculations. Many times the Markow Property cannot be guaranteed in real world decision problem which poses practical problems when wanting to employ RL-methods. Also, we note that conventional RL needs to be augmented by additional mechanisms, if one wants to employ it to more complex, for example time and space-continuous, systems. These aspects shall not be discussed here, but see Reynolds (2002); Santos and Touzet (1999a,b).
\chapter{Tabularne rešitve}
\thispagestyle{fancy}
\section{Dinamično programiranje}
\section{Napovedovanje - vrednost stanja}
\subsection{Monte Carlo metode}
\subsection{Učenje na podlagi časovne razlike - TD(0)} \label{subsection:TD0Prediction}
\subsection{Združitev metod - TD($\lambda$)}
\section{Krmiljenje - vrednost dejanja}
\subsection{Monte Carlo metode}
\subsection{Učenje na podlagi časovne razlike - TD(0)} \label{subsection:TD0Control}
\subsection{Združitev metod - TD($\lambda$)}

\chapter{Posploševanje in funkcijska aproksimacija}
%%%%% PONG GUY HAD MUCH BETTER SUCCESS WITH LINEAR OUPUT NEURON INSTEAD OF OTHER THINGIE.
\thispagestyle{fancy}
\section{Umetne nevronske mreže} %rename nevrološke mreže in nevronske mreže to umetne nevronske mreže
% overfitting -- early stopping and weight decay
\section{Metode gradient descent}
% instead just say what was implemented to generalize Hex?
%\section{Napovedovanje - vrednost stanja}

%\section{Krmiljenje - vrednost dejanja}

% compare size of ann to our brain

\chapter{Namizna igra Hex}
\thispagestyle{fancy}
\section{Ozadje}
\section{Učenje}
\section{Rezultati}
% graphs, errors

\chapter{Zaključek}
\thispagestyle{fancy}
Iz okrepitvenega učenja so se razvili solidni matematični temelji in impresivne aplikacije. Računska študija okrepitvenega učenja je sedaj obsežna, z aktivnimi raziskovalci na raznolikih disciplinah kot so psihologija, teorija krmiljenja (angl. control theory), operacijske raziskave (angl. operations research), umetna inteligenca in nevroznanost. Posebej pomembne so zveze z optimalnim nadzorom in dinamičnim programiranjem. Celoten problem učenja iz interakcije za dosego ciljev ni še zdaleč rešen, vendar se je naše razumevanje na tem področju bistveno izboljšalo. Sedaj lahko postavimo sestavne ideje kot so učenje na podlagi časovne razlike, dinamično programiranje in funkcijske aproksimacije skladno s celotnim problemom.

Eden večjih trendov katerih je okrepitveno učenje deležno je večji stik med umetno inteligenco in ostalimi inženirskimi disciplinami. Nedolgo nazaj se je umetno inteligenco smatralo kot popolnoma ločeno od teorije nadzora in statistiko~\cite{ReinforcementLearningAnIntroduction}. Imelo je opravka z logiko in simboli, ne pa s števili. Umetna inteligenca so bili obširni LISP programi, ne linearna algebra, diferencialne enačbe ali statistika. V zadnjih desetletjih se je ta pogled spremenil. Moderni raziskovalci umetne inteligence sprejemajo statistične in nadzorne algoritme kot pomembne konkurenčne metode ali pa enostavno kot orodja. Prej prezrta področja med umetno inteligenco in konvencionalnega inženirstva so sedaj med najbolj aktivnimi, vključno z nevronskimi mrežami, pametnim nadzorom in okrepitvenim učenjem. V okrepitvenem učenju se ideje optimalne teorije nadora in stohastične aproksimacije razširijo za nasloviti širše in bolj ambiciozne cilje umetne inteligence.

% while table td converges, function approximation td is tricky and not yet fully understood:
% 3.3 Scaling up in Tutorial survey and Recent Advances says something about ANN failures
% td with function approximation convergence
% td with non-linear function approximation does not converge
% some recent advances from Sutton: http://webdocs.cs.ualberta.ca/~sutton/publications.html

% Convergent Temporal-Difference Learning with Arbitrary Smooth Function Approximation
% We introduce the first temporal-difference learning algorithms that converge with smooth value function approximators, such as neural networks. Conventional temporal-difference (TD) methods, such as TD(λ), Q-learning and Sarsa have been used successfully with function approximation in many applications. How- ever, it is well known that off-policy sampling, as well as nonlinear function ap- proximation, can cause these algorithms to become unstable (i.e., the parameters of the approximator may diverge). Sutton et al. (2009a, 2009b) solved the prob- lem of off-policy learning with linear TD algorithms by introducing a new objec- tive function, related to the Bellman error, and algorithms that perform stochastic gradient-descent on this function. These methods can be viewed as natural gener- alizations to previous TD methods, as they converge to the same limit points when used with linear function approximation methods. We generalize this work to non- linear function approximation. We present a Bellman error objective function and two gradient-descent TD algorithms that optimize it. We prove the asymptotic almost-sure convergence of both algorithms, for any finite Markov decision pro- cess and any smooth value function approximator, to a locally optimal solution. The algorithms are incremental and the computational complexity per time step scales linearly with the number of parameters of the approximator. Empirical re- sults obtained in the game of Go demonstrate the algorithms’ effectiveness.
% In this paper, we solved a long-standing open problem in reinforcement learning, by establishing a family of temporal-difference learning algorithms that converge with arbitrary differentiable func- tion approximators (including neural networks). The algorithms perform gradient descent on a nat- ural objective function, the projected Bellman error. The local optima of this function coincide with solutions that could be obtained by TD(0). Of course, TD(0) need not converge with non-linear function approximation. Our algorithms are on-line, incremental and their computational cost per update is linear in the number of parameters. Our theoretical results guarantee convergence to a local optimum, under standard technical assumptions. Local optimality is the best one can hope for, since nonlinear function approximation creates non-convex optimization problems. The early empirical results obtained for computer Go are very promising. However, more practical experience with these algorithms is needed. We are currently working on extensions of these algorithms using eligibility traces, and on using them for solving control problems.

%A general gradient algorithm for temporal-difference prediction learning with eligibility traces
%ABSTRACT: A new family of gradient temporal-difference learning algorithms have recently been introduced by Sutton, Maei and others in which function approximation is much more straightforward. In this paper, we introduce the GQ(λ) algorithm which can be seen as extension of that work to a more general setting including eligibility traces and off-policy learning of temporally abstract predictions. These extensions bring us closer to the ultimate goal of this work—the development of a universal prediction learning algorithm suitable for learning experientially grounded knowledge of the world. Eligibility traces are essential to this goal because they bridge the temporal gaps in cause and effect when experience is processed at a temporally fine resolution. Temporally abstract predictions are also essential as the means for representing abstract, higher-level knowledge about courses of action, or options. GQ(λ) can be thought of as an extension of Q-learning. We extend existing convergence results for policy evaluation to this setting and carry out a forward-view/backward-view analysis to derive and prove the validity of the new algorithm.
%The GQ(λ) algorithm, which has been introduced in this paper, incorporates varying eligibility traces and option-conditional probabilities for policy evalu- ation. To derive GQ(λ), we carried out a forward- view/backward-view analysis. We extended the exist- ing convergence results to show that GQ(λ) is guaran- teed to converge to the TD(λ) fixed-point. GQ(λ) is a general gradient TD method for off-policy learning and as such can be seen as extension of Q-learning. GQ(λ) is able to learn about temporally abstract predictions, which makes it suitable to use for learning experientially grounded knowledge. In addition, GQ(λ) is online, in- cremental and its computational complexity scales only linearly with the size of features. Thus, it is suitable for large-scale applications. Our work, however, is limited to policy evaluation. Interesting future works is to ex- tend GQ(λ) for control problems and gather extensive empirical data on large-scale real-world applications.

Ray Kurzweil, inventor in futurist, je, v svoji nefiktivni knjigi ``The Singularity Is Near: When Humans Transcend Biology''~\cite{TheSingularityIsNear}, opisal svoj zakon o pospeševanju donosov, ki napoveduje eksponentno povečanje v tehnologijah kot so računalništvo, genetika, nanotehnologija, robotika in umetna inteligenca. Predvideva, da bo do okrog leta 2020 obstajal računalnik za tisoč ameriških dolarjev, ki bo imel računsko zmožnost posnemati človeško inteligenco. Po tem, pričakuje, da bo tehnologija optičnega zajemanja človeških možganov pripomogla k učinkovitemu modelu človeške inteligence do okrog 2025. Ta dva elementa bosta omogočala računalnikom opraviti Turingov preizkus do leta 2029. In do zgodnjih 2030 bo količina nebiološkega računanja prekoračilo zmožnost vse žive biološke inteligence človeštva. Končno eksponentno povečanje v računski zmožnosti bo privedlo do dogodka Singularnosti -- močno in moteče preoblikovanje v človeški sposobnosti -- leta 2045.

% deepmind
%\cite{PlayingAtariWithDeepReinforcementLearning}

%We present the first deep learning model to successfully learn control policies directly from high-dimensional sensory input using reinforcement learning. The model is a convolutional neural network, trained with a variant of Q-learning, whose input is raw pixels and whose output is a value function estimating future rewards. We apply our method to seven Atari 2600 games from the Arcade Learning Environment, with no adjustment of the architecture or learning algorithm. We find that it outperforms all previous approaches on six of the games and surpasses a human expert on three of them.

%http://www.theguardian.com/technology/shortcuts/2014/jan/28/demis-hassabis-15-facts-deepmind-technologies-founder-google

%http://www.technologyreview.com/news/524026/is-google-cornering-the-market-on-deep-learning/

\begin{thebibliography}{99}
\thispagestyle{fancy}
% Primeri:
% [1] Prvi A. Avtor, Drugi B. Avtor in Tretji C. Avtor, Naslov članka, Naslov revije 34 (2007), 24-56.
% [1] D. Marušič, R. Scapellato in B. Zgrablić, On quasiprimitive pqr-graphs, Algebra Colloq. 4 (1995), 295–314.
% [2] Prvi A. Avtor, Naslov knjige, druga izdaja. Založba, Ljubljana, 2008.
% [2] W. T. Tutte, Connnectivity in graphs, University of Toronto Press, Toronto, 1966.
% [3] Prvi A. Avtor in Drugi B. Avtor, Naslov poglavja v knjigi, v: Prvi urednik, Drugi urednik (ur.), Naslov knjige, Založba, Ljubljana, 1998, 2542–2603.
% [3] R. K. Guy, The decline and fall of Zarankiewicz's theorem, v: F. Harary (ur.), Proof Techniques in Graph Theory, Academic Press, New York, London, 1969, 63–69.

% Check capitalization of titles :\.

%\bibitem{ScienceAndHumanBehavior}
%B. F. Skinner
%\newblock{\em Science and Human Behavior},
%\newblock{Simon and Schuster, 1953}

\bibitem{SomeStudiesInMachineLearningUsingTheGameOfCheckers}
A. L. Samuel,
\newblock{\em Some Studies In Machine Learning Using the Game of Checkers},
\newblock{IBM Journal on Research and Development, 1959.}

\bibitem{TDANNForStrategicControlProblems}
A. Persson,
\newblock{\em Using Temporal Difference Methods In Combination With Artificial Neural Networks to Solve Strategic Control Problems},
\newblock{KTH Numerical Analysis and Computer Science, Royal Institute of Technology, Stockholm, Sweden, 2004.}

\bibitem{ComputationalModelsOfClassicalConditioningAComparativeStudy}
C. Balkenius, J. Morén,
\newblock{\em Computational Models of Classical Conditioning: A Comparative Study},
\newblock{From animals to animats 5: proceedings of the fifth international conference on simulation of adaptive behavior. MIT Press/Bradford Books: Cambridge, MA, 1998.}

\bibitem{AMathematicalTheoryOfCommunication}
C. E. Shannon,
\newblock{\em A Mathematical Theory of Communication},
\newblock{Bell Sys. Tech. Journal, vol. 27, 1948.}

\bibitem{IntroductionToPsychology}
C. Stangor,
\newblock{\em Introduction to Psychology},
\newblock{MIT Press, Cambridge, MA, 2011.}

\bibitem{MusicDiscriminationByPigeons}
D. Porter, A. Neuringer,
\newblock{\em Music Discrimination By Pigeons},
\newblock{Journal of Experimental Psychology: Animal Behavior Processes 10.2, 1984.}

\bibitem{AnimalIntelligence1}
E. L. Thorndike,
\newblock{\em Animal Intelligence: An Experimental Study of the Associative Processes In Animals},
\newblock{Psychological Monographs: General and Applied 2.4, 1898.}

\bibitem{AnimalIntelligence2}
E. L. Thorndike,
\newblock{\em Animal Intelligence: Experimental Studies},
\newblock{The Journal of Nervous and Mental Disease 39.5, 1912.}

\bibitem{PlayingRiskAversiveGoOnALargeBoardUsingLocalNeuralNetworkPositionEvaluationFunctions}
G. Markkula,
\newblock{\em Playing risk aversive go on a large board using local neural network position evaluation functions},
\newblock{Department of physical resource theory and complex systems group, Chalmers university of technology Göteborg, 2004.}

\bibitem{PracticalIssuesInTemporalDifferenceLearning}
G. Tesauro,
\newblock{\em Practical issues in temporal difference learning},
\newblock{Machine Learning 4, 1992.}

\bibitem{ConditionedReflexes}
I. Pavlov,
\newblock{\em Conditioned Reflexes},
\newblock{Courier Dover Publications, 2003.}

\bibitem{PavlovianConditioningItsNotWhatYouThinkItIs}
R. A. Rescorla,
\newblock{\em Pavlovian Conditioning: It's Not What You Think It Is},
\newblock{American Psychologist, 1988.}

\bibitem{TheSingularityIsNear}
R. Kurzweil,
\newblock{\em The Singularity Is Near: When Humans Transcend Biology},
\newblock{Penguin, 2005.}

\bibitem{LearningToPredictByTheMethodsOfTemporalDifference}
R. S. Sutton,
\newblock{\em Learning to predict by the methods of temporal difference},
\newblock{Machine Learning 3, 1988.}

\bibitem{ReinforcementLearningAnIntroduction}
R. S. Sutton, A. G. Barto,
\newblock{\em Reinforcement Learning: An Introduction},
\newblock{MIT Press, Cambridge, MA, 1998.}

\bibitem{LearningAndBehaviorAContemporarySynthesis}
M. E. Bouton,
\newblock{\em Learning and Behavior: A Contemporary Synthesis},
\newblock{Sinauer Associates, 2007.}

\bibitem{StrategyAcquisitionForTheGameOthelloBasedOnReinforcementLearning}
M. Ito, T. Yoshioka, S. Ishii,
\newblock{\em Strategy acquisition for the game “Othello” based on reinforcement learning},
\newblock{Technical report, Nara Institute of Science and Techology, 1998.}

\bibitem{LearningToEvaluateGoPositionsViaTemporalDifferenceMethods}
N. N. Schraudolf, P. Dayan, T. Sejnowski,
\newblock{\em Learning to evaluate go positions via temporal difference methods},
\newblock{Vol. 62 of Studies in fuzziness and soft computing, Springer Verlag, 2001.}

\bibitem{TheRoleOfExplorationInLearningControl}
S. B. Thrun,
\newblock{\em The Role of Exploration in Learning Control},
\newblock{Department of Computer Science, Carnegie-Mellon University, 1992.}

\bibitem{CognitionEvolutionAndBehavior}
S. J. Shettleworth,
\newblock{\em Cognition, Evolution and Behavior},
\newblock{Oxford University Press, 2009.}

\bibitem{ACollectionOfDefinitionsOfIntelligence}
S. Legg, M. Hutter,
\newblock{\em A Collection of Definitions of Intelligence},
\newblock{Frontiers in Aritificial Intelligence and Applications, 2007.}

\bibitem{PigeonsDiscriminationOfPaintings}
S. Watanabe, J. Sakamoto, M. Wakita,
\newblock{\em Pigeons' Discrimination of Paintings by Monet and Picasso},
\newblock{Journal of the experimental analysis of behavior 63.2, 1995.}

\bibitem{PsychologyAStudentFriendlyApproach}
T. L. Brink,
\newblock{\em Psychology: A Student Friendly Approach},
\newblock{San Bernardino Community College, 2008.}

\bibitem{PlayingAtariWithDeepReinforcementLearning}
V. Mnih, K. Kavukcuoglu, D. Silver, A. Graves, I. Antonoglou, D. Wierstra, M. Riedmiller,
\newblock{\em Playing Atari with Deep Reinforcement Learning},
\newblock{arXiv preprint arXiv:1312.5602, 2013.}

\end{thebibliography}

%\pagestyle{fancyplain}
%\vspace*{\fill}
%     \begin{center}
%          \bf{\Huge{Priloge}}
%     \end{center}
%\vspace*{\fill}
%\thispagestyle{fancy}
%\addcontentsline{toc}{chapter}{Priloge}
%\appendix
%\appendices{A Naslov prve priloge}
%\chapter{Naslov prve priloge}
%\thispagestyle{fancy}
%Tu dodamo prvo prilogo.
%\appendices{B Naslov druge priloge}
%\chapter{Naslov druge priloge}
%\thispagestyle{fancy}
%Tu dodamo drugo prilogo.

\end{document}