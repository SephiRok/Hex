\documentclass[a4paper, oneside, 12pt]{article}
\usepackage[slovene]{babel}
\usepackage[utf8]{inputenc}
\usepackage[left=3cm, top=3cm, right=2.5cm, bottom=2.5cm]{geometry}
\usepackage{fancyhdr}

\pagestyle{fancy}
\fontfamily{timesnewroman}
\linespread{1.25}

\lhead{\footnotesize Breulj R. Učenje iz interakcije\\UP FAMNIT, 2013}
\rhead{\thepage}
\cfoot{}

% I was looking for a way to learn without prior knowledge of the problem. A universal learner.

\begin{document}
\begin{titlepage}
\begin{center}
\begin{large}
UNIVERZA NA PRIMORSKEM\\
FAKULTETA ZA MATEMATIKO, NARAVOSLOVJE IN\\
INFORMACIJSKE TEHNOLOGIJE\\[6cm]
\end{large}
\end{center}

\begin{center}
Zaključna naloga\\
{\large Učenje iz interakcije}\\
Learning from interaction\\[6cm]
\end{center}

\noindent
Ime in priimek: Rok Breulj\\
Študijski program: Računalništvo in informatika\\
Mentor: doc. dr. Peter Rogelj\\

\vfill
\begin{center}
{\large Koper, Avgust 2013}
\end{center}
\end{titlepage}

\section*{Ključna dokumentacijska informacija}
\newpage

\section*{Key words documentation}
\newpage

\section*{Zahvala}
\newpage

%\section*{Kazala}
%\subsection*{Kazalo vsebine}
\tableofcontents
\newpage

%\subsection*{Kazalo preglednic}
\listoftables
\newpage

%\subsection*{Kazalo slik in grafikonov}
\listoffigures
\newpage

%\subsection*{Kazalo prilog}

%\subsection*{Seznam kratic}

\section{Uvod}
\subsection{Okrepitveno učenje}
\subsection{Primeri}
\subsection{Elementi okrepitvenega učenja}
\newpage

\section{Problem}
\subsection{Ocenjevanje povratne informacije}
\subsection{Celoten problem okrepitvenega učenja}
\newpage

\section{Rešitve}
\subsection{Dinamično programiranje}
\subsection{Predvidevanje - vrednost stanja}
\subsubsection{Monte Carlo metode}
\subsubsection{Učenje na podlagi časovne razlike - TD(0)}
\subsubsection{Združitev metod - TD($\lambda$)}
\subsection{Krmiljenje - vrednost dejanja}
\subsubsection{Monte Carlo metode}
\subsubsection{Učenje na podlagi časovne razlike - TD(0)}
\subsubsection{Združitev metod - TD($\lambda$)}
\newpage

\section{Posploševanje in funkcijska aproksimacija}
\subsection{Predvidevanje - vrednost stanja}
\subsection{Krmiljenje - vrednost dejanja}
\newpage

\section{Učenje na namizni igri Hex}
\subsection{Ozadje}
\subsection{Implementacija}
\newpage

\section{Zaključek}
\newpage

\section{Literatura}
\newpage

\section{Priloge}
\newpage

\end{document}