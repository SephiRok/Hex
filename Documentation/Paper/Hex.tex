\documentclass[a4paper, oneside, 12pt]{article}
\usepackage[slovene]{babel}
\usepackage[utf8]{inputenc}
\usepackage[left=3cm, top=3cm, right=2.5cm, bottom=2.5cm]{geometry}
\usepackage{fancyhdr}

\pagestyle{fancy}
\fontfamily{timesnewroman}
\linespread{1.25}

% Strojno učenje: učenje iz interakcije
% Strojno učenje: učenje z interakcijo
% Strojno učenje: okrepitveno učenje
% Učenje z interakcijo z okoljem

\lhead{\footnotesize Breulj R. Učenje iz interakcije\\UP FAMNIT, 2013}
\rhead{\thepage}
\cfoot{}

% I was looking for a way to learn without prior knowledge of the problem. A universal learner.
% Reinforcement learning is one way -- trial and error interaction with the environment
% Observation (of the correct approach) is another, faster way of getting it right
% Teaching is the fastest way -- supervised learning

\begin{document}
\begin{titlepage}
\begin{center}
\begin{large}
UNIVERZA NA PRIMORSKEM\\
FAKULTETA ZA MATEMATIKO, NARAVOSLOVJE IN\\
INFORMACIJSKE TEHNOLOGIJE\\[6cm]
\end{large}
\end{center}

\begin{center}
Zaključna naloga\\
{\large Učenje iz interakcije}\\
Learning from interaction\\[6cm]
\end{center}

\noindent
Ime in priimek: Rok Breulj\\
Študijski program: Računalništvo in informatika\\
Mentor: doc. dr. Peter Rogelj\\

\vfill
\begin{center}
{\large Koper, September 2013}
\end{center}
\end{titlepage}

\section*{Ključna dokumentacijska informacija}
\newpage

\section*{Key words documentation}
\newpage

\section*{Zahvala}
\newpage

%\section*{Kazala}
%\subsection*{Kazalo vsebine}
\tableofcontents
\newpage

%\subsection*{Kazalo preglednic}
\listoftables
\newpage

%\subsection*{Kazalo slik in grafikonov}
\listoffigures
\newpage

%\subsection*{Kazalo prilog}

%\subsection*{Seznam kratic}

\section{Uvod}
% Meh, just translated. Rewrite.
%children playing with blocks
%matching shapes
Ideja učenja iz interakcije z našim okoljem je ena od prvih, ki nam pride na misel, ko razmišljamo o naravi učenja. Ko se dojenček igra, maha z rokami ali gleda naokoli nima izrecnega učitelja, ima pa neposredno senzomotorično povezavo z okoljem. Uporaba te povezave proizvede ogromno informacije o vzrokih in učinkih, o posledicah dejanj in načinih kako doseči cilje. Skozi naše življenje so takšne interakcije nedvoumno velik izvir znanja o našem okolju in samim sebi. Ko se učimo voziti avto ali pogovarjati, se zavedamo kako se okolje odziva na naša dejanja in iščemo način kako vplivati na rezultat z našim vedenjem.

What's in this work.

\subsection{Okrepitveno učenje}
Okrepitveno učenje (angl. reinforcement learning) je učenje kaj narediti, kako izbirati dejanje, da povečamo številčni nagrajevalni signal. Učencu niso nikoli predstavljena pravilna ali optimalna dejanja kot pri večini oblik strojnega učenja. Katera dejanja prinesejo največjo nagrado mora sam odkriti s poizkušanjem. Skozi interakcijo z okoljem se uči posledic svojih dejanj. V najbolj zanimivih in težavnih primerih imajo dejanja vpliv ne le na takojšnjo nagrado ampak tudi na naslednji položaj in posledične nagrade. Te dve karakteristiki, iskanje s poizkušanjem in zamudne nagrade, so dve najpomembnejši lastnosti okrepitvenega učenja.

Okrepitveno učenje ni definirano s karakterističnimi metodami učenja temveč kot karakterizacija problema učenja. Katerokoli metodo primerno za rešitev problema smatramo kot metodo okrepitvenega učenja. Celoten problem okrepitvenega učenja je predstavljen na strani \pageref{section:problem}. Osnovna ideja je zajeti najpomembnejše vidike realnega problema s katerim se sooča učenec (angl. learning agent) pri interakciji s svojim okoljem za dosego cilja. Takšen učenec mora imeti nekakšna čutila za pridobivanje informacij o stanju okolja in mora biti sposoben vplivati na to stanje z dejanji. Imeti mora tudi cilj ali pa cilje, ki se nanašajo na stanje okolja. Namen opisa problema je predstaviti te vidike, čutenje, dejanje in cilj, v najenostavnejši obliki brez poenostavljenja.

Okrepitveno učenje se razlikuje od nadzorovanega učenja (angl. supervised learning) v tem, da nima izobraženega zunanjega nadzornika, ki predloži učencu primere in rezultate. Nadzorovano učenje je pomemben tip učenja vendar ni primerno za učenje iz interakcije. V interaktivnih problemih je velikokrat nepraktično pridobiti primere želenega vedenja, ki so pravilni in hkrati predstavljajo vsa stanja v katerih mora učenec delovati. V neznanem okolju, kjer bi si predstavljali, da je učenje najbolj koristno, se mora učenec učiti iz svojih izkušenj [[[[[[[[[[[1]]]]]]]]]]].

Eden od izzivov okrepitvenega učenja, ki jih ne najdemo v ostalih oblikah strojnega učenja, je kompromis med ___raziskovanjem___ (angl. exploration) in izkoriščanjem (angl. exploitation). 


%One of the challenges that arise in reinforcement learning and not in other kinds of learning is the trade-off between exploration and exploitation. To obtain a lot of reward, a reinforcement learning agent must prefer actions that it has tried in the past and found to be effective in producing reward. But to discover such actions, it has to try actions that it has not selected before. The agent has to exploit what it already knows in order to obtain reward, but it also has to explore in order to make better action selections in the future. The dilemma is that neither exploration nor exploitation can be pursued exclusively without failing at the task. The agent must try a variety of actions and progressively favor those that appear to be best. On a stochastic task, each action must be tried many times to gain a reliable estimate its expected reward. The exploration-exploitation dilemma has been intensively studied by mathematicians for many decades (see Chapter 2). For now, we simply note that the entire issue of balancing exploration and exploitation does not even arise in supervised learning as it is usually defined.



% Further, there is a focus on on-line performance, which involves finding a balance between exploration (of uncharted territory) and exploitation (of current knowledge).

% The reinforcement signal that the RL-agent receives is a numerical reward, which encodes the success of an action's outcome, and the agent seeks to learn to select actions that maximize the accumulated reward over time. (The use of the term reward is used here in a neutral fashion and does not imply any pleasure, hedonic impact or other psychological interpretations.)

% Ask Peter for input on copying too much. :'(

\subsection{Primeri}
\subsection{Elementi okrepitvenega učenja}
\newpage

\section{Problem}
\label{section:problem}
\subsection{Ocenjevanje povratne informacije}
\subsection{Celoten problem okrepitvenega učenja}
\newpage

\section{Rešitve}
\subsection{Dinamično programiranje}
\subsection{Predvidevanje - vrednost stanja}
\subsubsection{Monte Carlo metode}
\subsubsection{Učenje na podlagi časovne razlike - TD(0)}
\subsubsection{Združitev metod - TD($\lambda$)}
\subsection{Krmiljenje - vrednost dejanja}
\subsubsection{Monte Carlo metode}
\subsubsection{Učenje na podlagi časovne razlike - TD(0)}
\subsubsection{Združitev metod - TD($\lambda$)}
\newpage

\section{Posploševanje in funkcijska aproksimacija}
\subsection{Predvidevanje - vrednost stanja}
\subsection{Krmiljenje - vrednost dejanja}
\newpage

\section{Učenje na namizni igri Hex}
\subsection{Ozadje}
\subsection{Implementacija}
\newpage

\section{Zaključek}
\newpage

\section{Literatura}
\newpage

\section{Priloge}
\newpage

\end{document}