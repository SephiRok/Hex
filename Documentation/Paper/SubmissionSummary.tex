\documentclass[a4paper, oneside, 12pt]{article}
\usepackage[slovene]{babel}
\usepackage[utf8]{inputenc}
\usepackage[left=3cm, top=3cm, right=2.5cm, bottom=2.5cm]{geometry}

\fontfamily{timesnewroman}
\linespread{1.25}

\begin{document}
\section*{Strojno učenje iz interakcije}
\subsection*{Kratka predstavitev zaključne naloge}
{\em Rok Breulj, UP FAMNIT junij 2014}
\newline
\newline
Učimo se skozi naše celotno življenje. En od osnovnih načinov učenja temelji na podlagi interakcije z okoljem. V računalništvu velikokrat radi odidemo po tej eksperimentalni poti, posebej ko verjamemo, da smo blizu rešitvi. Ampak ni potrebno pogledati tako daleč kot je računalništvo. Že kot otroci, ko mahamo z rokami in nogami ter gledamo naokoli, nimamo izrecnega učitelja, imamo pa neposredno senzomotorično povezavo z našo okolico. S svojim vedenjem vplivamo na okolico in naša čutila izkoriščamo za pridobitev ogromne količine podatkov o vzrokih in učinkih, o posledicah dejanj in kako doseči cilje.

Skozi implementacijo namizne igre Hex v nalogi raziščem okrepitveno učenje (angl. reinforcement learning), področje strojnega učenja, ki se ukvarja z vprašanjem kako se vesti v neznanem okolju, da bi povečali številčni nagrajevalni signal. Okrepitveno učenje se razlikuje od nadzorovanega učenja v tem, da nam niso nikoli prikazana pravilna dejanja ali pa napačna popravljena. Poudarek je tudi na zmogljivosti učenja med izvedbo, katero pelje do iskanja ravnovesja med raziskovanjem neznanih stanj in izkoriščanjem obstoječega znanja. Ukvarja se s celotnim problemom učenja iz interakcije z neznanim okoljem.
\end{document}